\documentclass[a4paper,12pt]{article}
\usepackage{amssymb}
\usepackage{graphicx}
\usepackage{hyperref}
\usepackage{blindtext}

\usepackage{color}
\usepackage{amsmath,geometry,makeidx}
\topmargin -1.2in
\textheight 11in
\oddsidemargin -0.45in
\evensidemargin 0in
\textwidth 7in
\footskip 0.5in
\flushbottom

\renewcommand\maketitle{%
	\setlength\parindent{0pt}%
	\begin{minipage}{\textwidth}
		\begin{minipage}{.15\textwidth}
			\includegraphics[width=1.5\linewidth]{OM-logo}
		\end{minipage}%
		\begin{minipage}{.7\textwidth}
			\centering
			Installation Sheet for OpenModelica\par
			OpenModelica Team\par
			FOSSEE, IIT Bombay
		\end{minipage}%
		\begin{minipage}{.15\textwidth}
			\raggedright
			\includegraphics[width=\linewidth]{FOSSEE-logo}
		\end{minipage}%
	\end{minipage}\vskip 2.5ex
	\par
	\hrule
}

\begin{document}

\maketitle

\section{The procedure to install OpenModelica on Windows OS}

\begin{enumerate}
\item To follow the installation procedure, you need to be connected to the internet.
\item Open your default web browser.
\item In the address bar, type the url: \url{https://www.openmodelica.org} and press Enter.
\item Go to ``DOWNLOAD" tab.
\item Select ``Windows".
\item From ``Offcial Release", Click on 1.18.0 (32bit/64bit)
\item From Parent Directory
\begin{enumerate}
	\item If you are using a 64-bit system:
	\begin{enumerate}
	\item click on 64bit.
	\item Click on OpenModelica-v1.18.0-64bit.exe.
	\item Save OpenModelica-v1.18.0-64bit.exe file.
	\end{enumerate}
	\item If you are using a 32-bit system:
	\begin{enumerate}
	\item click on 32bit.
	\item Click on OpenModelica-v1.18.0-32bit.exe.
	\item Save OpenModelica-v1.18.0-32bit.exe file.
	\end{enumerate}
\end{enumerate}
\item Right Click on the downloaded file and select Run as Administrator.
\item In Installation Pop-up window, click on Next.
\item Choose the Destination Folder and click on Next.
\item Click on Install.
\item Click on Next.
\item Click on Finish.
\item OpenModelica is successfully installed.
\end{enumerate}

\section{The procedure to install OpenModelica on Linux OS}

\begin{enumerate}
	
\item To follow the installation procedure, you need to be connected to the internet.
\item Open terminal and type: sudo apt-get update
\item Type your system password.
\item Type: echo ``deb http://build.openmodelica.org/apt `lsb\_release -cs' release"
\item In the terminal, type: sudo gedit /etc/apt/sources.list
\item A new gedit file named ``sourses.list" appears.
\item At the end of the page, type: deb http://build.openmodelica.org/apt focal release
\item Press CTRL + S and close the file
\item Type: wget -q http://build.openmodelica.org/apt/openmodelica.asc -O- $|$ sudo apt-key add -
\item It will show ``OK".
\item Open a new terminal window
\item Type: sudo apt update
\item Type: sudo apt install openmodelica
\item OpenModelica is successfully installed.
%\item Type: for PKG in `apt-cache search "omlib-.*" $|$ cut -d" " -f1`; do sudo apt-get install -y "\$PKG"; done (It installs optional Modelica libraries)	
\end{enumerate}

\section{Checking the installation}

\begin{enumerate}
	\item To check the software installation, please follow these steps:
	\item For Windows: Go to ``OpenModelica Connection Editor", right click on it and select ``Run as administrator"
	\newline For Linux: Open a command terminal by pressing Ctrl+Alt+T and type ``OMEdit".
	\item When opening OMEdit, it will ask to choose one of the versions of MSL (Modelica Standard Library). Select MSL v3.2.3 and proceed.
	\item Expand the ``Modelica" library from Libraries Browser.
	\item Click on the arrow head to the left of ``Thermal" library.
	\item Under ``Thermal", expand ``HeatTransfer" library.
	\item Now expand the ``Examples" library.
	\item Double click on ``TwoMasses" class.
	\item Now click on ``Simulate" button on the toolbar.
	\item In the new window, go to the ``Variables Browser" at the right.
	\item Expand the ``conduction" variable.
	\item Click on the check box against dT variable.
	\item We will get a plot between time and dT.
\end{enumerate}

\end{document}
